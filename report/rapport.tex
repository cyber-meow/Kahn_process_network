%%
% Copyright (c) 2017, Pascal Wagler;  
% Copyright (c) 2014--2017, John MacFarlane
% 
% All rights reserved.
% 
% Redistribution and use in source and binary forms, with or without 
% modification, are permitted provided that the following conditions 
% are met:
% 
% - Redistributions of source code must retain the above copyright 
% notice, this list of conditions and the following disclaimer.
% 
% - Redistributions in binary form must reproduce the above copyright 
% notice, this list of conditions and the following disclaimer in the 
% documentation and/or other materials provided with the distribution.
% 
% - Neither the name of John MacFarlane nor the names of other 
% contributors may be used to endorse or promote products derived 
% from this software without specific prior written permission.
% 
% THIS SOFTWARE IS PROVIDED BY THE COPYRIGHT HOLDERS AND CONTRIBUTORS 
% "AS IS" AND ANY EXPRESS OR IMPLIED WARRANTIES, INCLUDING, BUT NOT 
% LIMITED TO, THE IMPLIED WARRANTIES OF MERCHANTABILITY AND FITNESS 
% FOR A PARTICULAR PURPOSE ARE DISCLAIMED. IN NO EVENT SHALL THE 
% COPYRIGHT OWNER OR CONTRIBUTORS BE LIABLE FOR ANY DIRECT, INDIRECT, 
% INCIDENTAL, SPECIAL, EXEMPLARY, OR CONSEQUENTIAL DAMAGES (INCLUDING,
% BUT NOT LIMITED TO, PROCUREMENT OF SUBSTITUTE GOODS OR SERVICES; 
% LOSS OF USE, DATA, OR PROFITS; OR BUSINESS INTERRUPTION) HOWEVER 
% CAUSED AND ON ANY THEORY OF LIABILITY, WHETHER IN CONTRACT, STRICT 
% LIABILITY, OR TORT (INCLUDING NEGLIGENCE OR OTHERWISE) ARISING IN 
% ANY WAY OUT OF THE USE OF THIS SOFTWARE, EVEN IF ADVISED OF THE 
% POSSIBILITY OF SUCH DAMAGE.
%%

\documentclass[]{scrartcl}
\usepackage{lmodern}
\usepackage{amssymb,amsmath}
\usepackage{ifxetex,ifluatex}
\usepackage{fixltx2e} % provides \textsubscript
\ifnum 0\ifxetex 1\fi\ifluatex 1\fi=0 % if pdftex
  \usepackage[T1]{fontenc}
  \usepackage[utf8]{inputenc}
\else % if luatex or xelatex
  \usepackage{unicode-math}
  \defaultfontfeatures{Ligatures=TeX,Scale=MatchLowercase}
\fi
% use upquote if available, for straight quotes in verbatim environments
\IfFileExists{upquote.sty}{\usepackage{upquote}}{}
% use microtype if available
\IfFileExists{microtype.sty}{%
\usepackage[]{microtype}
\UseMicrotypeSet[protrusion]{basicmath} % disable protrusion for tt fonts
}{}
\PassOptionsToPackage{hyphens}{url} % url is loaded by hyperref
\usepackage[unicode=true]{hyperref}
\hypersetup{
            pdfborder={0 0 0},
            breaklinks=true}
\urlstyle{same}  % don't use monospace font for urls
\usepackage{listings}
\newcommand{\passthrough}[1]{#1}
\IfFileExists{parskip.sty}{%
\usepackage{parskip}
}{% else
\setlength{\parindent}{0pt}
\setlength{\parskip}{6pt plus 2pt minus 1pt}
}
\setlength{\emergencystretch}{3em}  % prevent overfull lines
\providecommand{\tightlist}{%
  \setlength{\itemsep}{0pt}\setlength{\parskip}{0pt}}
\setcounter{secnumdepth}{0}
% Redefines (sub)paragraphs to behave more like sections
\ifx\paragraph\undefined\else
\let\oldparagraph\paragraph
\renewcommand{\paragraph}[1]{\oldparagraph{#1}\mbox{}}
\fi
\ifx\subparagraph\undefined\else
\let\oldsubparagraph\subparagraph
\renewcommand{\subparagraph}[1]{\oldsubparagraph{#1}\mbox{}}
\fi

% set default figure placement to htbp
\makeatletter
\def\fps@figure{htbp}
\makeatother


\date{}



%%
%% added
%%

%
% language
%

\usepackage[ngerman]{babel}


%
% colors
%
\usepackage[usenames, dvipsnames, svgnames, table]{xcolor}

%
% listing colors
%
\definecolor{listing-background}{rgb}{0.97,0.97,0.97}
\definecolor{listing-rule}{HTML}{B3B2B3}
\definecolor{listing-numbers}{HTML}{B3B2B3}
\definecolor{listing-text-color}{HTML}{000000}
\definecolor{listing-keyword}{HTML}{435489}
\definecolor{listing-identifier}{HTML}{435489}
\definecolor{listing-string}{HTML}{00999a}
\definecolor{listing-comment}{HTML}{8e8e8e}
\definecolor{listing-javadoc-comment}{HTML}{006CA9}

%\definecolor{listing-background}{rgb}{0.97,0.97,0.97}
%\definecolor{listing-rule}{HTML}{B3B2B3}
%\definecolor{listing-numbers}{HTML}{B3B2B3}
%\definecolor{listing-text-color}{HTML}{000000}
%\definecolor{listing-keyword}{HTML}{D8006B}
%\definecolor{listing-identifier}{HTML}{000000}
%\definecolor{listing-string}{HTML}{006CA9}
%\definecolor{listing-comment}{rgb}{0.25,0.5,0.35}
%\definecolor{listing-javadoc-comment}{HTML}{006CA9}

%
% TOC depth and 
% section numbering depth
%
\setcounter{tocdepth}{3}

%
% line spacing
%
\usepackage{setspace}
\setstretch{1.2}

%
% break urls
%
\PassOptionsToPackage{hyphens}{url}

%
% When using babel or polyglossia with biblatex, loading csquotes is recommended 
% to ensure that quoted texts are typeset according to the rules of your main language.
%
\usepackage{csquotes}

%
% paper size and 
% margins
%
\usepackage[a4paper,margin=2.5cm, left=3cm,includehead=true,includefoot=true,centering]{geometry}

%
% captions
%
\usepackage[font={small,it}]{caption}
\newcommand{\imglabel}[1]{\textbf{\textit{(#1)}}}

%
% sans-serif font as default font for text
% Source Sans Pro for headings
% Source Code Pro for monospace text
%
\renewcommand*{\familydefault}{\sfdefault}
\usepackage{sourcesanspro}
\usepackage{sourcecodepro}

%
% heading font
%
\newcommand*{\heading}{\fontfamily{\sfdefault}\selectfont}

%
% heading color
%
\usepackage{relsize}
\usepackage{sectsty}
\definecolor{almostblack}{RGB}{40,40,40}
\allsectionsfont{\sffamily\color{almostblack}}

%
% variables for title and author
%
\usepackage{titling}
\title{}
\author{}

%
% environment for boxes
%
%\usepackage{framed}

%
% tables
%
\usepackage{booktabs} % needed for midrule
\usepackage{tabularx}
\renewcommand{\arraystretch}{1.6} % table spacing


%
% remove paragraph indention
%
\setlength{\parindent}{0pt}
\setlength{\parskip}{6pt plus 2pt minus 1pt}
\setlength{\emergencystretch}{3em}  % prevent overfull lines

%
%
% Listings
%
%

\lstdefinestyle{eisvogel_listing_style}{
  language=java,
  numbers=left,
  backgroundcolor=\color{listing-background},
  basicstyle=\color{listing-text-color}\small\ttfamily{}, % print whole listing small
  xleftmargin=0.8em, % 2.8 with line numbers
  breaklines=true,
  frame=single,
  framesep=0.6mm,
  rulecolor=\color{listing-rule},
  frameround=ffff,
  framexleftmargin=0.4em, % 2.4 with line numbers | 0.4 without them
  tabsize=4, %width of tabs
  numberstyle=\color{listing-numbers},
  aboveskip=1.0em,
  keywordstyle=\color{listing-keyword}\bfseries, % underlined bold black keywords
  classoffset=0,
  sensitive=true,
  identifierstyle=\color{listing-identifier}, % nothing happens
  commentstyle=\color{listing-comment}, % white comments
  morecomment=[s][\color{listing-javadoc-comment}]{/**}{*/},
  stringstyle=\color{listing-string}, % typewriter type for strings
  showstringspaces=false, % no special string spaces
  escapeinside={/*@}{@*/}, % for comments
  literate=
  {á}{{\'a}}1 {é}{{\'e}}1 {í}{{\'i}}1 {ó}{{\'o}}1 {ú}{{\'u}}1
  {Á}{{\'A}}1 {É}{{\'E}}1 {Í}{{\'I}}1 {Ó}{{\'O}}1 {Ú}{{\'U}}1
  {à}{{\`a}}1 {è}{{\'e}}1 {ì}{{\`i}}1 {ò}{{\`o}}1 {ù}{{\`u}}1
  {À}{{\`A}}1 {È}{{\'E}}1 {Ì}{{\`I}}1 {Ò}{{\`O}}1 {Ù}{{\`U}}1
  {ä}{{\"a}}1 {ë}{{\"e}}1 {ï}{{\"i}}1 {ö}{{\"o}}1 {ü}{{\"u}}1
  {Ä}{{\"A}}1 {Ë}{{\"E}}1 {Ï}{{\"I}}1 {Ö}{{\"O}}1 {Ü}{{\"U}}1
  {â}{{\^a}}1 {ê}{{\^e}}1 {î}{{\^i}}1 {ô}{{\^o}}1 {û}{{\^u}}1
  {Â}{{\^A}}1 {Ê}{{\^E}}1 {Î}{{\^I}}1 {Ô}{{\^O}}1 {Û}{{\^U}}1
  {œ}{{\oe}}1 {Œ}{{\OE}}1 {æ}{{\ae}}1 {Æ}{{\AE}}1 {ß}{{\ss}}1
  {ç}{{\c c}}1 {Ç}{{\c C}}1 {ø}{{\o}}1 {å}{{\r a}}1 {Å}{{\r A}}1
  {€}{{\EUR}}1 {£}{{\pounds}}1 {«}{{\guillemotleft}}1
  {»}{{\guillemotright}}1 {ñ}{{\~n}}1 {Ñ}{{\~N}}1 {¿}{{?`}}1
}
\lstset{style=eisvogel_listing_style}

\lstdefinelanguage{XML}
{
  morestring=[b]",
  moredelim=[s][\bfseries\color{listing-keyword}]{<}{\ },
  moredelim=[s][\bfseries\color{listing-keyword}]{</}{>},
  moredelim=[l][\bfseries\color{listing-keyword}]{/>},
  moredelim=[l][\bfseries\color{listing-keyword}]{>},
  morecomment=[s]{<?}{?>},
  morecomment=[s]{<!--}{-->},
  commentstyle=\color{listing-comment},
  stringstyle=\color{listing-string},
  identifierstyle=\color{listing-identifier}
}

%
% header and footer
%
\usepackage{fancyhdr}
\pagestyle{fancy}
\fancyhead{}
\fancyfoot{}
\lhead{}
\chead{}
\rhead{}
\lfoot{}
\cfoot{}
\rfoot{\thepage}
\renewcommand{\headrulewidth}{0.4pt}
\renewcommand{\footrulewidth}{0.4pt}

%%
%% end added
%%



\begin{document}

\section{Projet Réseaux de Kahn}\label{projet-ruxe9seaux-de-kahn}

Yu-Guan Hsieh \& Téo Sanchez

\subsection{Résumé du projet}\label{ruxe9sumuxe9-du-projet}

Le but de ce projet est de réaliser différentes implémentations de
réseaux de Kahn à partir d'une interface donnée sous forme monadique (ce
qui nous permet de manipuler des langages fonctionnels purs avec des
traits impératifs).

Les réseaux de Kahn (Kahn Process Networks en anglais), sont un modèle
de calcul distribué entre plusieurs processus communiquant entre eux par
des files. Ce réseau a un comportement déterministe si les processus qui
le composent sont déterministes.

Les différentes implémentations sont :

\begin{enumerate}
\def\labelenumi{\arabic{enumi}.}
\itemsep1pt\parskip0pt\parsep0pt
\item
  Une implémentation utilisant la bibliothèque de threads d'OCaml
  (donnée dans l'énoncé)
\item
  Une implémentation utilisant la bibliothèque de thread Lwt. Cette
  implémentation a servi de référence car Lwt contient dèjà des
  fonctions caractéristiques de la communication inter-processus.
\item
  Un implémentation utilisant des processus Unix communiquant entre eux
  avec des pipes grâce à la bibliothèque Unix d'OCaml.
\item
  Une implémentation séquentielle où le parallélisme est simulé (inspiré
  de l'implémentation par continuation de l'article \emph{A Poor Man's
  Concurrency Monad}).
\item
  Une implémentation distribuée sur le réseaux utilisant des sockets de
  la bibliothèque Unix d'OCaml
\end{enumerate}

\subsection{Les différentes
implémentations}\label{les-diffuxe9rentes-impluxe9mentations}

\subsubsection{Threads d'OCaml (fournie) :
\lstinline!kahn_th.ml!}\label{threads-docaml-fournie-kahnux5fth.ml}

Dans cette implémentation, les processus sont représentés par des
fonctions de type \lstinline!unit -> 'a!. C'est une promesse qui nous
rendra une valeur de type \lstinline!'a! une fois que la fonction
\lstinline!run! est exécutée. Chaque processus individuel vit dans son
propre thread.

\subsubsection{Threads de Lwt :
\lstinline!kahn_lwt.ml!}\label{threads-de-lwt-kahnux5flwt.ml}

Ici, les processus sont des threads de type \lstinline!'a Lwt.t!, et la
plupart des fonctions requises sont déjà implémentés dans le module Lwt
: return, bind, doco (\lstinline!Lwt.join!), run, get
(\lstinline!Lwt_stream.next!). L'implémentation est triviale.

\subsubsection{Processus lourd du module Unix d'OCaml :
\lstinline!kahn_proc.ml!}\label{processus-lourd-du-module-unix-docaml-kahnux5fproc.ml}

De manière analogue aux threads d'OCaml, les processus sont une fonction
\lstinline!unit -> 'a!. On utilise des pipes pour les canaux, couplés
avec des mutex. Ces derniers préviennent des problèmes liés aux
ressources partagées entre les processus, qui sont des portions de codes
que l'on appellent zones critiques. Ils font parties des techniques
dites d'exclusion mutuelles, n'autorisant l'accès à la zone critique par
un seul processus à la fois.

Les fonctions get et put utilisent le module \lstinline!Marshal! qui
permet d'encoder n'importe quelle structures de données en séquences de
bytes, afin d'être\\envoyés sur les canaux pour être ensuite décodés par
le processus destinataire.

Enfin, la fonction \lstinline!doco! qui prend une liste de processus et
les exécute. Avec \lstinline!Unix.waitpid!, le processus entier est
bloquant: tous les processus de la liste doivent finir avant de passer à
la prochaine étape. La fonction renvoie \lstinline!unit! une fois que
tous les processus de la liste ont été exécutées.

\subsubsection{Implémentation séquentielle, parallélisme simulé :
\lstinline!kahn_seq.ml!}\label{impluxe9mentation-suxe9quentielle-paralluxe9lisme-simuluxe9-kahnux5fseq.ml}

On distingue le \enquote{vrai} parallélisme (où un ordinateurs
multi-cœurs lance des processus sur plusieurs de ses processeurs) du
parallélisme à temps partagé, où les threads s'exécutent sur un même
processeur. Dans ce dernier cas, le parallélisme est simulé. Cette idée
est incarnée par deux implémentations possibles :

La première consiste à traduire directement le code Haskell issu de
l'article \enquote{\emph{A Poor Man's Concurrency Monad}} en OCaml à
quelques nuances près, même si la structure reste la même. Elle utilise
la méthode d'\emph{entrelacement} (\emph{interleaving} en anglais) c'est
à dire que le processeur va exécuter le début d'un thread, avant de le
suspendre pour donner la main à un autre thread. Afin de relancer les
thread suspendus au même endroit, on doit avoir accès à son
\enquote{futur} appelé généralement sa \emph{continuation}. Cette
implémentation monadique invoque des fonctions qui prennent une
\emph{continuation} comme premier arguement.

Ainsi, les processus sont divisés en tranches appelés actions, et ces
mêmes actions renvoient leur futur qui sont elles mêmes des actions :

\begin{lstlisting}
type action =
  | Stop    (* Fin du processus*)
  | Action of (unit -> action)
  | Doco of action list
\end{lstlisting}

Un processus est alors de type \lstinline!('a -> action) -> action!.

Une autre implémentation possible pour cette section est d'utiliser un
type \lstinline!('a -> unit) -> unit! comme proposé dans l'énoncé. Cela
requiert alors d'avoir une structure de donnée globale qui stocke ce
qu'il reste à faire après chaque étape de l'exécution d'un processus.

Nous avons choisi ici la première implémentation issue de l'article
\enquote{A poor Man's Concurrency Monad}.

\subsubsection{Implémentation distribuée sur le réseau :
\lstinline!kahn_network.ml!}\label{impluxe9mentation-distribuuxe9e-sur-le-ruxe9seau-kahnux5fnetwork.ml}

\paragraph{Déscription}\label{duxe9scription}

Cette implémentation a pour objectif de distribuer les processus sur
plusieurs ordinateurs et communiquant à travers le réseaux via des
sockets. On utilise les sockets implémentés dans le module Unix d'OCaml
et les données sont transféréss avec le module \lstinline!Marshal!.

On distingue:

\begin{itemize}
\itemsep1pt\parskip0pt\parsep0pt
\item
  Les processus qui sont du type \lstinline!CSet.t -> 'a * CSet.t!, avec
  \lstinline!CSet! une structure de donnée créée avec le module
  \lstinline!Set! d'OCaml, et qui permet de stocker les canaux ouverts à
  un moment donné. Un processus prend l'ensemble des canaux ouverts et
  renvoie les canaux ouvers après l'exécution du processus.
\item
  Un canal est défini par son numéro de port \lstinline!port_num : int!,
  le nom de son ordinateur hôte \lstinline!host : string!, et le champs
  \lstinline!sock : (sock *   sock_kind) option! qui défini le sens de
  la communication (qui est le producteur et qui est le consommateur) et
  stocke le type abstrait de canal (qui se compose de
  \lstinline!out_channel! et \lstinline!in_channel! d'OCaml) si la
  connection a été établie.
\item
  Une file des ordinateurs disponibles \lstinline!computer_queue!, créée
  à partir du fichier \emph{network.config} où l'on doit écrire la liste
  des ordinateurs que l'on souhaite utiliser.
\end{itemize}

La fonction \lstinline!new_channel! créée deux threads. L'un communique
avec les producteurs et l'autre avec les consommateurs au travers de
sockets. Les deux threads communiquent également entre eux par un pipe :
Le premier thread lit les données dans les sockets entre lui et les
producteurs et les stocke dans le pipe tandis que le second thread lit
les données qui sont mis dans le pipe et les mets dans les sockets vers
les consommateurs quand il reçoit une requête de sa part
(\lstinline!GET!: mettre une valeur dans le canal; \lstinline!GETEND!:
la fin de communication).

La fonction \lstinline!send_processes! assure la distribution des
processus sur les différents ordinateurs de \lstinline!computer_queue!
et établit les connections grâce aux sockets (à travers la fonction
\lstinline!easy_connect!). Puis dans la fonction \lstinline!doco!, grâce
à l'utilisation de la fonction \lstinline!Unix.select!, on surveille que
les processus fils s'exécutent, et dans le cas contraire, il faut
redistribuer le processus.

La fonction put prend un élément \lstinline!v! à envoyer, et l'ensemble
des canaux ouverts, et vérifie s'il existe un canal concret dans la
socket (elle le crée sinon). Elle utilise \lstinline!Marshal! pour
envoyer la valeur dans le canal et renvoie \lstinline!unit! et le nouvel
ensemble des canaux ouverts. La fonction \lstinline!get! procède de
manière analogue sauf qu'elle renvoie une valeur.

Les fonctions \lstinline!commu_with_send! et \lstinline!commu_with_recv!
gèrent le remplacement des producteurs et des consommateurs
respectivement, en appelant un nouveau client quand la connection est
rompue ou lorsque le processus lui renvoie le signal \lstinline!PutEnd!
ou \lstinline!GetEnd! signifiant la fin de la communication.

\paragraph{Utilisation}\label{utilisation}

Cette implémentation ne fonctionne qu'avec OCaml \textgreater{}= 4.03.0
de façon génerale (dans des cas particuliers, comme le fichier
\lstinline!int_printer_network.ml!, ça pourrait marcher avec OCaml
4.02.0).

Il y a plusieurs options dans la ligne de commande ,

\begin{itemize}
\itemsep1pt\parskip0pt\parsep0pt
\item
  \lstinline!-wait! : doit être utilisé sur tous les ordinateurs du
  réseaux à l'exception de l'ordinateur principal qui devra être lancé à
  la fin pour démarre le programme.
\item
  \lstinline!-port! : spécifie le port à écouter (1024 par défaut)
\item
  \lstinline!-config! : permet de spécifier le nom du fichier contenant
  la liste des ordinateurs à utiliser (par défaut, ce fichier est
  \emph{network.config}).
\end{itemize}

Le fichier \emph{network.config} se présente sous le format suivant :

\begin{lstlisting}
Computer1 [port1]
Computer2 [port2]
...
\end{lstlisting}

Le même ordinateur peut apparaître plusieurs fois si les ports sont
différents.

\paragraph{Améliorations possibles}\label{amuxe9liorations-possibles}

Lorsque la communication avec un ordinateur est interrompue de manière
inopinée, le programme peut ne plus continuer correctement puisque le
processus distribué interrompu sera redémarré depuis le début. Par
exemple, avec l'exemple de base de la génération d'entier, si la
fonction d'affichage sur la sortie standard est interrompu, elle sera
redémarré et continuera d'afficher les entiers qu'il reçoit. Le résultat
ne sera donc pas altéré. A l'inverse, s'il s'agit de la fonction qui
génère les entiers, après l'arrêt de ce processus, le programme
affichera les entiers depuis le début.

Pour règler ce problème, on a essayé de demander le processus fils de
renvoyer son futur pour chaque \lstinline!bind! exécutée. Il y a des
modifications à faire, et surtout le type d'un processus ne resterait
plus le même mais ça n'a pas abouti à cause du module
\lstinline!Marshal! qui ne prend pas en charge le type abstrait.

En OCaml avec le module \lstinline!Arg! le parsing de la ligne de
commande ne peut s'effectuer que dans un seul endroit, ce qui nous pose
de problème car on a besoin que ça soit fait plusieurs fois (une fois
pour le foncteur \lstinline!Choose_impl!, une fois pour la fonction
\lstinline!run! dans l'implémentation de réseau et enfin une fois dans
le programme utilisateur). Pour contourner cette difficulté, on a choisi
d'utiliser la fonction \lstinline!Arg.parse_argv! au lieu de
\lstinline!Arg.parse! et on modifie directement la valeur de
\lstinline!Sys.argv!. Il y a quelques défauts de cette solution:

\begin{enumerate}
\def\labelenumi{\arabic{enumi}.}
\item
  Le tableau \lstinline!Sys.argv! peut contenir des chaînes de
  caractères vides après le parsing. L'utilisateur de la bibliothèque
  doit les négliger.
\item
  Il nous manque un message complèt indiquant tous ces spécifications
  qui peut s'afficher quelque parts quand il y en a besoin (par exemple
  avec la commande \lstinline!--help!).
\end{enumerate}

\textbf{Anecdote:} En OCaml 4.03.0 et 4.04.0, dans le module
\lstinline!Arg! avec la fonction \lstinline!Arg.parse!, pour une option
de ligne de commande qui prend un seul argument (c'est ainsi le cas pour
\lstinline!-port! dans notre programme), le message d'erreur peut
s'afficher trois fois si aucun argument est donné à cette option.

\subsection{Exemples d'applications}\label{exemples-dapplications}

\subsubsection{Exemples basiques}\label{exemples-basiques}

\begin{itemize}
\itemsep1pt\parskip0pt\parsep0pt
\item
  \lstinline!put_get_test.ml! : un processus met l'entier 2 dans un
  canal et l'autre le lit et l'affiche. Faire \lstinline!make put_get!
  pour générer ce programme
\item
  \lstinline!int_printer.ml! : Un processus génère une suite croissante
  infini d'entier, tandis que le second les récupère et les affiche dans
  la sortie standard.
\item
  \lstinline!alter_print.ml! : Deux processus écrivent et lisent
  alternativement une suite d'entier dans deux canaux.
\end{itemize}

\subsubsection{Crible d'Ératosthène}\label{crible-duxe9ratosthuxe8ne}

Il s'agit d'un procédé permettant de trouver tous les nombres premiers
inférieurs à un certain entier naturel donné. Le fichier
\lstinline!sieve_Erathostene.ml! contient l'algorithme décrit à la page
9 de l'article \emph{Coroutines and Networks of Parallel Processes} de
Gilles Kahn et David MacQueen. Le nombre de processus utilisés dans cet
algorithme n'est pas borné et donc ne fonctionne pas bien avec
l'implémentation distribuée par le réseau.

\subsubsection{Tracé de l'ensemble de
Mandelbrot}\label{tracuxe9-de-lensemble-de-mandelbrot}

L'ensemble de Mandelbrot est une fractale définie comme l'ensemble des
points c du plan complexe pour lesquels la suite des nombres complexes
définie par récurrence par

z0 = 0 zn+1 = zn\^{}2 +c

est bornée.

Le fichier \lstinline!Mandelbrot.ml! affiche l'ensemble sur
{[}-2;2{]}x{[}-1.5;1.5{]}. L'image est divisée en plusieurs zones et le
calcul de chaque zone est pris en charge par un processus. On peut
spécifier la taille de l'image, le nombre de zones et le nombre
d'itérations pour chaque point.

\subsubsection{Pong}\label{pong}

L'implémentation de pong n'utilise que la version réseau des KPN, un
ordinateur lance le programme avec l'option \lstinline!-wait! tandis que
l'autre peut choisir les paramètres de jeu.

\subsubsection{K-moyennes}\label{k-moyennes}

Il s'agit d'un algorithme de partitionnement des données d'un problème
d'optimisation combinatoire. Le fichier d'entrée contient sur chaque
ligne un point dont les coordonnées sont séparées par des espaces. Le
nombre de partition k, le nombre d'itérations i, et le nombre d'ouvriers
p (de processus en parallèle) peuvent être données en arguments. Les
centres des partitions sont ensuite calculées et écrits dans un fichier
de sortie spécifié par l'option \lstinline!-o!.

\subsection{Conclusion et
perspectives}\label{conclusion-et-perspectives}

map reduce picture n picture

\end{document}