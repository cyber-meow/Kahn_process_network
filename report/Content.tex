
\title{Projet Réseaux de Kahn\vspace{-0.5em}}

\author{Yu-Guan Hsieh \& Téo Sanchez}
\date{\today}

\maketitle

\section{Résumé du projet}\label{ruxe9sumuxe9-du-projet}

Le but de ce projet est de réaliser différentes implémentations de
réseaux de Kahn à partir d'une interface donnée sous forme monadique (ce
qui nous permet de manipuler des langages fonctionnels purs avec des
traits impératifs).

Les réseaux de Kahn (Kahn Process Networks en anglais), sont un modèle
de calcul distribué entre plusieurs processus communiquant entre eux par
des files. Ce réseau a un comportement déterministe si les processus qui
le composent sont déterministes.

Les différentes implémentations sont :

\begin{enumerate}
\def\labelenumi{\arabic{enumi}.}
\itemsep1pt\parskip0pt\parsep0pt
\item
  Une implémentation utilisant la bibliothèque de threads d'OCaml
  (donnée dans l'énoncé)
\item
  Une implémentation utilisant la bibliothèque de thread \texttt{Lwt}. Cette
    implémentation a servi de référence car \texttt{Lwt} contient dèjà des
  fonctions caractéristiques de la communication inter-processus.
\item
  Un implémentation utilisant des processus Unix communiquant entre eux
  avec des pipes grâce à la bibliothèque \texttt{Unix} d'OCaml.
\item
  Une implémentation séquentielle où le parallélisme est simulé (inspiré
  de l'article \emph{A Poor Man's Concurrency Monad}).
\item
  Une implémentation distribuée sur le réseau utilisant des sockets de
    la bibliothèque \texttt{Unix} d'OCaml
\end{enumerate}

\section{Les différentes
implémentations}\label{les-diffuxe9rentes-impluxe9mentations}

\subsection{Threads d'OCaml (fournie) :
\texttt{kahn\_th.ml}}\label{threads-docaml-fournie-kahnux5fth.ml}

Dans cette implémentation, les processus sont représentés par des
fonctions de type \texttt{unit -\textgreater{} 'a}. C'est une promesse
qui nous rendra une valeur de type \texttt{'a} une fois que la fonction
\texttt{run} est exécutée. Chaque processus individuel vit dans son
propre thread.

\subsection{Threads de \texttt{Lwt} :
\texttt{kahn\_lwt.ml}}\label{threads-de-lwt-kahnux5flwt.ml}

Ici, les processus sont des threads de type \texttt{'a Lwt.t}, et la
plupart des fonctions requises sont déjà implémentés dans le module 
\texttt{Lwt}: \texttt{return}, \texttt{bind}, \texttt{doco} 
(\texttt{Lwt.join}), \texttt{run}, \texttt{get}
(\texttt{Lwt\_stream.next}). L'implémentation est triviale.

\subsection{Processus lourd :
\texttt{kahn\_proc.ml}}\label{processus-lourd-du-module-unix-docaml-kahnux5fproc.ml}

De manière analogue aux threads d'OCaml, les processus sont une fonction
\texttt{unit -\textgreater{} 'a}. On utilise des pipes pour les canaux,
couplés avec des mutex. Ces derniers préviennent des problèmes liés aux
ressources partagées entre les processus, qui sont des portions de codes
que l'on appelle zones critiques. Ils font parties des techniques
dites d'exclusion mutuelles, n'autorisant l'accès à la zone critique par
un seul processus à la fois.

Les fonctions get et put utilisent le module \texttt{Marshal} qui permet
d'encoder n'importe quelle structures de données en séquences de bytes,
afin d'être envoyés sur les canaux pour être ensuite décodés par le
processus destinataire.

Enfin, la fonction \texttt{doco} prend une liste de processus et les
exécute. Avec \texttt{Unix.waitpid}, le processus entier est bloquant:
tous les processus de la liste doivent finir avant de passer à la
prochaine étape. La fonction renvoie \texttt{unit} une fois que tous les
processus de la liste ont été exécutés.

\subsection{Implémentation séquentielle :
\texttt{kahn\_seq.ml}}\label{impluxe9mentation-suxe9quentielle-paralluxe9lisme-simuluxe9-kahnux5fseq.ml}

On distingue le ``vrai'' parallélisme (où un ordinateurs multi-cœurs
lance des processus sur plusieurs de ses processeurs) du parallélisme à
temps partagé, où les threads s'exécutent sur un même processeur. Dans
ce dernier cas, le parallélisme est simulé. Cette idée est incarnée par
deux implémentations possibles :

La première consiste à traduire directement le code Haskell issu de
l'article \emph{A Poor Man's Concurrency Monad} en OCaml à quelques
nuances près, même si la structure reste la même. Elle utilise la
méthode d'\emph{entrelacement} (\emph{interleaving} en anglais) c'est à
dire que le processeur va exécuter le début d'un thread, avant de le
suspendre pour donner la main à un autre thread. Afin de relancer les
thread suspendus au même endroit, on doit avoir accès à son ``futur''
appelé généralement sa \emph{continuation}. Cette implémentation
monadique invoque des fonctions qui prennent une \emph{continuation}
comme premier arguement.

Ainsi, les processus sont divisés en tranches appelés actions, et ces
mêmes actions renvoient leur futur qui sont elles mêmes des actions :

\begin{Shaded}
\begin{Highlighting}[]
\KeywordTok{type} \NormalTok{action =}
  \NormalTok{| }\DataTypeTok{Stop}
  \NormalTok{| }\DataTypeTok{Action} \KeywordTok{of} \NormalTok{(}\DataTypeTok{unit} \NormalTok{-> action)}
  \NormalTok{| }\DataTypeTok{Doco} \KeywordTok{of} \NormalTok{action }\DataTypeTok{list}
\end{Highlighting}
\end{Shaded}

Un processus est alors de type
\texttt{('a -\textgreater{} action) -\textgreater{} action}.

Une autre implémentation possible pour cette section est d'utiliser un
type \texttt{('a -\textgreater{} unit) -\textgreater{} unit} comme
proposé dans l'énoncé. Cela requiert alors d'avoir une structure de
donnée globale qui stocke ce qu'il reste à faire après chaque étape de
l'exécution d'un processus.

Nous avons choisi ici la première implémentation issue de l'article \emph{A
poor Man's Concurrency Monad}.

\subsection{Implémentation distribuée sur le réseau :
\texttt{kahn\_network.ml}}\label{impluxe9mentation-distribuuxe9e-sur-le-ruxe9seau-kahnux5fnetwork.ml}

\subsubsection{Déscription}\label{duxe9scription}

Cette implémentation a pour objectif de distribuer les processus sur
plusieurs ordinateurs et communiquant à travers le réseau via des
sockets. On utilise les sockets implémentés dans le module \texttt{Unix}
d'OCaml et les données sont transféréss avec le module \texttt{Marshal}.

On distingue:

\begin{itemize}
\itemsep1pt\parskip0pt\parsep0pt
\item
  Les processus qui sont du type
  \texttt{CSet.t -\textgreater{} 'a * CSet.t}, avec \texttt{CSet} une
  structure de donnée créée avec le module \texttt{Set} d'OCaml, et qui
  permet de stocker les canaux ouverts à un moment donné. Un processus
  prend l'ensemble des canaux ouverts et renvoie les canaux ouvers après
  l'exécution du processus.
\item
  Un canal est défini par son numéro de port \texttt{port\_num:int},
  le nom de son ordinateur hôte \texttt{host:string}, et le champs
  \texttt{sock:(sock *   sock\_kind) option} qui défini le sens de la
  communication (qui est le producteur et qui est le consommateur) et
  stocke le type abstrait de canal (qui se compose de
  \texttt{out\_channel} et \texttt{in\_channel} d'OCaml) si la
  connection a été établie.
\item
  Une file des ordinateurs disponibles \texttt{computer\_queue}, créée à
  partir du fichier \emph{network.config} où l'on doit écrire la liste
  des ordinateurs que l'on souhaite utiliser.
\end{itemize}

La fonction \texttt{new\_channel} crée deux threads. L'un communique
avec les producteurs et l'autre avec les consommateurs au travers de
sockets. Les deux threads communiquent également entre eux par un pipe :
Le premier thread lit les données dans les sockets entre lui et les
producteurs et les stocke dans le pipe tandis que le second thread lit
les données qui sont mis dans le pipe et les mets dans les sockets vers
les consommateurs quand il reçoit une requête de sa part (\texttt{GET}:
mettre une valeur dans le canal; \texttt{GETEND}: la fin de
communication).

La fonction \texttt{send\_processes} assure la distribution des
processus sur les différents ordinateurs de \texttt{computer\_queue} et
établit les connections grâce aux sockets (à travers la fonction
\texttt{easy\_connect}). Puis dans la fonction \texttt{doco}, grâce à
l'utilisation de la fonction \texttt{Unix.select}, on surveille que les
processus fils s'exécutent, et dans le cas contraire, il faut
redistribuer le processus.

On précise aussi que sur chaque ordinatuer (ou encore mieux, sur chaque
noeud de réseau car un programme peut être lancé plusieurs fois sur une
même machine) il y a un thread qui se charge d'accepter les
distributions de processus et de créer des nouveau threads dans lesquels
ces processus seront exécutés. Le choix d'utiliser des processus légers
autorise ainsi l'existence des variables partagées au sein d'un même
exemplaire de programme.

La fonction \texttt{put} prend un élément \texttt{v} à envoyer, et l'ensemble 
des canaux ouverts, et vérifie s'il existe un canal concret dans la socket
(elle le crée sinon). Elle utilise \texttt{Marshal} pour envoyer la
valeur dans le canal et renvoie \texttt{unit} et le nouvel ensemble des
canaux ouverts. La fonction \texttt{get} procède de manière analogue
sauf qu'elle renvoie une valeur.

Les fonctions \texttt{commu\_with\_send} et \texttt{commu\_with\_recv}
gèrent le remplacement des producteurs et des consommateurs
respectivement, en appelant un nouveau client quand la connection est
rompue ou lorsque le processus lui renvoie le signal \texttt{PutEnd} ou
\texttt{GetEnd} signifiant la fin de la communication.

\subsubsection{Utilisation}\label{utilisation}

Cette implémentation ne fonctionne qu'avec OCaml \textgreater{}= 4.03.0
de façon génerale (dans des cas particuliers, comme le fichier
\texttt{int\_printer\_network.ml}, ça pourrait marcher avec OCaml
4.02.0).

Il y a plusieurs options dans la ligne de commande ,

\begin{itemize}
\itemsep1pt\parskip0pt\parsep0pt
\item
  \texttt{-wait} : doit être utilisé sur tous les ordinateurs du réseau
  à l'exception de l'ordinateur principal qui devra être lancé à la fin
  pour démarrer le programme.
\item
  \texttt{-port} : spécifie le port à écouter (1024 par défaut)
\item
  \texttt{-config} : permet de spécifier le nom du fichier contenant la
  liste des ordinateurs à utiliser (par défaut, ce fichier est
  \emph{network.config}).
\end{itemize}

Le fichier \emph{network.config} se présente sous le format suivant :

\begin{verbatim}
Computer1 [port1]
Computer2 [port2]
...
\end{verbatim}

Le même ordinateur peut apparaître plusieurs fois si les ports sont
différents.

\subsubsection{D'autres problèmes
techniques}\label{dautres-probluxe8mes-techniques}

Lorsque la communication avec un ordinateur est interrompue de manière
inopinée, le programme peut ne plus continuer correctement puisque le
processus distribué interrompu sera redémarré depuis le début. Par
exemple, avec l'exemple de base de la génération d'entier, si la
fonction d'affichage sur la sortie standard est interrompu, elle sera
redémarré et continuera d'afficher les entiers qu'il reçoit. Le résultat
ne sera donc pas altéré. A l'inverse, s'il s'agit de la fonction qui
génère les entiers, après l'arrêt de ce processus, le programme
affichera les entiers depuis le début.

Pour règler ce problème, on a essayé de demander le processus fils de
renvoyer son futur pour chaque \texttt{bind} exécutée. Il y a des
modifications à faire, et surtout le type d'un processus ne resterait
plus le même, mais ça n'a pas abouti à cause du module \texttt{Marshal}
qui ne prend pas en charge le type abstrait (même si on n'a pas pu
identifier d'où vient ce type abstrait en jeu empêchant notre code de
fonctionner).

Afin que les communications à travers des canaux soient plus robustes,
des questions sont à poser. Comment vérifier que le message envoyé a été
bien reçu (et par la bonne personne)? Comment affirmer que les message
reçus viennent effectivment du programme en question (et que les
messages sont corrects)? Les idées comme des clés d'authentification
peuvent être considérées.

Il y a encore un autre petit soucis dans notre réalisation de la version
réseau: l'incompatibilité entre la fonction \texttt{Unix.select} et le
module \texttt{Graphics} d'OCaml. Un bloc \texttt{try... with...} a été
adopté face à cette gêne.

\section{Exemples d'applications}\label{exemples-dapplications}

\subsection{Exemples basiques}\label{exemples-basiques}

\begin{itemize}
\itemsep1pt\parskip0pt\parsep0pt
\item
  \texttt{put\_get\_test.ml} : un processus met l'entier 2 dans un canal
  et l'autre le lit et l'affiche.
\item
  \texttt{int\_printer.ml} : Un processus génère une suite croissante
  infini d'entier, tandis que le second les récupère et les affiche dans
  la sortie standard.
\item
  \texttt{alter\_print.ml} : Deux processus écrivent et lisent
  alternativement une suite d'entiers dans deux canaux.
\end{itemize}

\subsection{Crible d'Ératosthène}\label{crible-duxe9ratosthuxe8ne}

Il s'agit d'un procédé permettant de trouver tous les nombres premiers
inférieurs à un certain entier naturel donné. Le fichier
\texttt{sieve\_Erathostene.ml} contient l'algorithme décrit à la page 9
de l'article \emph{Coroutines and Networks of Parallel Processes} de
Gilles Kahn et David MacQueen. Le nombre de processus utilisés dans cet
algorithme n'est pas borné et donc ne fonctionne pas bien avec
l'implémentation distribuée par le réseau.

\subsection{Tracé de l'ensemble de
Mandelbrot}\label{tracuxe9-de-lensemble-de-mandelbrot}

L'ensemble de Mandelbrot est une fractale définie comme l'ensemble des
points $c$ du plan complexe pour lesquels la suite des nombres complexes
définie par récurrence par

\begin{align*}
  z_0 &= 0\\
  z_{n+1} &= z_n^2 + c
\end{align*}

est bornée.

Le fichier \texttt{Mandelbrot.ml} affiche cet ensemble sur la région 
{[}-2;2{]} x {[}-1.5;1.5{]}. L'image est divisée en plusieurs zones et le 
calcul de chaque zone est pris en charge par un processus. On peut spécifier 
la taille de l'image, le nombre de zones et le nombre d'itérations pour
chaque point.

\subsection{Pong}\label{pong}

L'implémentation de pong n'utilise que la version réseau des KPN, un
ordinateur lance le programme avec l'option \texttt{-wait} tandis que
l'autre peut choisir les paramètres de jeu.

\subsection{K-moyennes}\label{k-moyennes}

Il s'agit d'un algorithme de partitionnement des données d'un problème
d'optimisation combinatoire. Le fichier d'entrée contient sur chaque
ligne un point dont les coordonnées sont séparées par des espaces. Le
nombre de partitions \texttt{k}, le nombre d'itérations \texttt{i}, le
nombre d'ouvriers \texttt{p} (de processus en parallèle), et le nombre
de fois à exécuter l'algorithme \texttt{t} peuvent être données en
arguments. Les centres des partitions sont ensuite calculées et écrits
dans un fichier de sortie spécifié par l'option \texttt{-o}. En plus, si
les entrées sont des points de dimension 2, on peut afficher le résultat
en utilisant l'option \texttt{-plot}.

\section{Divers}\label{divers}

\subsection{Parsing de la ligne de
commande}\label{parsing-de-la-ligne-de-commande}

En OCaml avec le module \texttt{Arg} le parsing de la ligne de commande
ne peut s'effectuer que dans un seul endroit, ce qui nous pose de
difficulté car on a besoin que ça soit fait plusieurs fois (une fois
pour le foncteur \texttt{Choose\_impl}, une fois pour la fonction
\texttt{run} dans l'implémentation de réseau et enfin une fois dans le
programme utilisateur). Pour contourner ce problème, on a choisi
d'utiliser la fonction \texttt{Arg.parse\_argv} au lieu de
\texttt{Arg.parse} et on modifie directement la valeur de
\texttt{Sys.argv}. Il y a quelques défauts de cette solution:

\begin{enumerate}
\def\labelenumi{\arabic{enumi}.}
\item
  Le tableau \texttt{Sys.argv} peut contenir des chaînes de caractères
  vides après le parsing. L'utilisateur de la bibliothèque doit les
  négliger.
\item
  Il nous manque un message complèt indiquant tous ces spécifications
  qui peut s'afficher quelque parts quand il y en a besoin (par exemple
  avec la commande \texttt{-{}-help}).
\end{enumerate}

\textbf{Anecdote:} En OCaml 4.03.0 et 4.04.0, dans le module
\texttt{Arg} avec la fonction \texttt{Arg.parse}, pour une option de
ligne de commande qui prend un seul argument (c'est ainsi le cas pour
\texttt{-port} dans notre programme), le message d'erreur peut
s'afficher trois fois si aucun argument est donné à cette option.

\subsection{MapReduce}\label{mapreduce}

On a voulu implémenter le modèle de MapReduce dans le cadre de réseau de
Kahn, mais c'est enfin abandonné dû à deux raisons principales:

\begin{enumerate}
\def\labelenumi{\arabic{enumi}.}
\item
  \textbf{MapReduce est indéterministe, alors que KPN est déterministe}\\[0.5em]
  Le modèle MR est indéterministe dans le sens qu'il n'y
  a pas un ordre prédéfini des exécutions de tâches. Les résultats des
  ouvriers sont traités \emph{immédiatement} par le patron dès qu'ils
  sont produits, ce qui est impossible dans un réseau de Kahn car on n'a
  pas de droit de tester si un canal est vide ou pas: une fois qu'on est
  bloqué, on est bloqué. Par conséquent, en modèlisant MapReduce par
  KPN, tout devient déterministe, ce qui montre une différence
  intrinsèque entre ces deux modèles.
\item
  \textbf{Le rôle de patron}\\[0.5em]
  Observons que dans le module
  \texttt{Functory}, le patron existe particulièrement pour effectuer un
  effet de bord, ce qui est gênant car quand on fait un \texttt{doco}
  dans les réseaux de Kahn, on aimerait bien que tous les processus
  soient purs (à éventuellement les opérations I/O près). Le fait que le
  patron peut effectuer un effet de bord impose qu'il vit dans le même
  ordinateur et le même processus Unix que le process qui a effectué la
  \texttt{doco}, ce que l'on n'a à priori pas de droit de controler au
  niveau d'un modèle abstrait tel que celui de réseau de Kahn.
\end{enumerate}

Les deux points ci-dessus ne nous empêchent pas d'implémenter une
interface MapReduce en utilisant notre bibliothèque de réseau de Kahn.
Pourtant, ils nous font remarquer des nuances éventuelles entre le vrai
MapReduce et un MapReduce qui est simulé par KPN.

